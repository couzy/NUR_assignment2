\section{Exercise 1}
In this section we make simple routines, which are usefull in the latter part of the assignment.
The first routine is the one to make a Poisson distribution. The poisson distribution is given by:
\begin{equation}
P(k,\lambda)=\frac{lambda^k\exp{-\lambda}}{k!}
\end{equation}
We generate the output for the following values of (\lambda,k)=(1,0),(5,10),(3,21),(2.6,40).
As the k-factorial goes very large, there is danger of overflow for this distribution. To counter this we set up a power of 10 for the factorial and divide each number bigger than 10, by 10 and add then one to our counter of our power of 10.
This lets us to work around for the overflow as some of the power gets mitigated of the k-factorial.

The second routine we set up is a Random Number Generator(RNG) for the 'continuous' range(0,1). This is done with a combination of a 64-bit XOR shift and after this, use an LCG to get a better pseudo RNG. For our RNG we set the initial seed at 23, so the answers are consistent in the future. 

\lstinputlisting{exercise1.py}

The result of the script is given by:
\lstinputlisting{exercise1.txt}

The RNG seems to work fine on the interval (0,1) the numbers are almost evenly distributed with small fluctuations, which are normal for any finite sample.
\begin{figure}[h]
   \centering
   \includegraphics[width=5cm]{plots/rng01.png}
      \caption{The  distribution of 1 million iterations of the RNG on the interval (0,1)}
   \end{figure}
