\section{Exercise 1}
In this section we make simple routines, which are usefull in the latter part of the assignment.

At first we set up is a Random Number Generator(RNG) for the 'continuous' range(0,1). This is done with a combination of a 64-bit XOR shift and after we use a Multiply with Carry on this 64-bit number and use an LCG to get a better pseudo RNG. For our RNG we set the initial seed at 23, so the answers are consistent in the future. 

\lstinputlisting{exercise1.py}


The result of the script is given by:
\lstinputlisting{exercise1.txt}

The RNG seems to work fine on the interval (0,1) the $10^6$ numbers are almost evenly distributed with small fluctuations, which are normal for any finite sample.
\begin{figure}[h]
   \centering
   \includegraphics[width=5cm]{plots/distribution.png}
      \caption{The distribution of 1 million iterations of the RNG on the interval (0,1)}
   \end{figure}
  
 

 For better judgement we plot for the first 1000 numbers plotted where number $x_i$ is plotted against number $x_{i+1}$, the distribution seems very random, there is some clustering but this is not rare. There is not clear pattern by eye in this RNG.
 
\begin{figure}[h]
   \centering
   \includegraphics[width=5cm]{plots/relative_distrng.png}
      \caption{The distribution of first 1000 $x_i$ vs $x_{i+1}$}
   \end{figure}
   
   
  We want to use there random numbers to change the uniform RNG to a Gaussian RNG, as in the future we want our numbers to be Gaussian distributed. To go from a uniform random sample to a Gaussian random sample, we use the Box-Muller method. For each pair of random numbers, which are generated from a uniform distribution on the interval $(0,1)$, this gives back 2 random numbers, with a Gaussian distribution with $\mu=0$ and $\sigma=1$. We translate and scale these numbers to get to a different $\mu$ and $\sigma$. We then plot 1000 numbers from our Box-Muller method, with $\mu=2.4,\sigma=3$ and overplot the a true Gaussian and compare the two distributions to each other, so if our Box-Muller method is valid.
  \begin{figure}[h]
   \centering
   \includegraphics[width=5cm]{plots/Box_muller.png}
      \caption{A histogram of our Box-Muller method compared to the true Gaussian, the lines are at $\pm4,\pm3,\pm2,\pm1 \sigma$
   \end{figure}
  
  The random numbers follow the Gaussian fairly well, this gives us confidende that the method is working correctly. However 
  
  
