\section{Exercise 7}
In this exercise we will construct a Quad-Tree. A Quad-tree is an algorithm that divides your 2-dimensional square into four smaller squares, where we keep making the squares smaller until a threshold is reached. This can be like a small number of particles in each square, so we can calculate the moments of the particles in each of the squares. This is generally done in the Barnes-Hut algorithm, to efficiently calculate the gravity interaction between a large set of particles and do not do in the naive $N^2$ way. We have given data which consists of 1200 particles, with x,y-coordinates.

For these particles we construct a class Quad-tree, a class nodes and a class points. In this we put our points in the root of the tree and then subdivide it into smaller nodes until the threshold of 12 particles is reached, at each node we calculate the n=0 moments. This is the amount of particles in each leaf. If we want to calculate the momentum in each node, we have to sum over the leafs to get to it. We use some of the code in the link \href{https://kpully.github.io/Quadtrees/}{https://kpully.github.io/Quadtrees/} as inspiration, as this makes the construction of the class a bit easier. After this we set rectangles at all of our nodes and plot the points inside of the rectangles. So the visualization of our quad-tree can be seen below in figure\ref{quadtree}

\begin{figure}[h]
   \centering
   \includegraphics[width=10cm]{plots/Quadtree.png}
      \caption{Our quad-tree implementation on the particles.}
    \label{quadtree}
\end{figure}

The quad-tree looks very nice, there seems to be 2 clusters of stars, where the number of particles is very dense. At these regions the tree needs to go down very far to reach the threshold of 12 particles. For other regions which are sparse only a few iterations made it reach the threshold.
We have then implemented a recursive search of the particle with index 100. We go down in our tree checking which branch we need to take to get to the right leaf, we do this recursively until we have gone to the final leaf, this is the leaf where the threshold of 2 particles have been reached and then we terminate the loop, during this we have appended the moments of all the nodes of the final leaf. Then we output the n=0 moment for that leaf and all the parent nodes.


The output of this exercise is given by:
\lstinputlisting{exercise7.txt}

\lstinputlisting{exercise7.py}