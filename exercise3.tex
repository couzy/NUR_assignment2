\section{Exercise 3}
This exercise will be about linear structure growth. We use linear pertubation theory for the early Universe. The main equation of the linear density perturbation $\delta $is the following:
\begin{equation}
\frac{\partial^2\delta}{\partial t^2}+2\frac{\dot{a}}{a}\frac{\partial\delta}{\delta t}=\frac{3}{2}\Omega_0H_0^3\frac{\delta}{a^3}
\end{equation}

Here $a$ is the scale factor of the Universe, $H_0$ the Hubble constant at the present and $\Omega_0$ the matter density parameter at $a=1$, in this equation we assume an Einstein de Sitter Universe and $\Omega_0=1$.

If we split our function of $\delta(x,t)$ in a spatial dependent part$\Delta(x)$ and a part which depends on time $D(t)$. If we focus now on the time dependent part, the partial differential equation above becomes:
\begin{equation}
\frac{d^2 D}{dt^2}+2\frac{\dot{a}}{a}\frac{dD}{d t}=\frac{3}{2}\Omega_0H_0^2\frac{D}{a^3}
\end{equation}

With our assumption of our Einstein-de Sitter Universe our scale factor $a$ has the following equation:
\begin{equation}
a=\left(\frac{3}{2}H_0t\right)^{2/3}
\end{equation}

This means that $\frac{\dot{a}}{a}=\frac{2}{3t}$ and $\frac{3}{2a^3}\Omega_0H_0^2=\frac{2}{3t^2}$

Applying this to our second order differential equation we get:
\begin{equation}
\frac{d^2 D}{dt^2}+\frac{4}{3t}\frac{dD}{d t}=\frac{2}{3t^2} D
\end{equation}

This is a second order ordinary differential equation.
Assume a solution of $t^{w}$. Then we get
\begin{equation}
    (w)(w-1)t^{w-2}+\frac{4w}{3t}t^{w-1}-\frac{2}{3t^{2}}t^{w}
\end{equation}
Multiply all factors by $t^2$.
\begin{equation}
    w(w-1)t^{w}+\frac{4w}{3}t^{w}-\frac{2}{3}t^{w}=0
\end{equation}
Now divide all factors by $t^{w}$
\begin{equation}
    w(w-1)+\frac{4w}{3}-\frac{2}{3}=0
\end{equation}
We then find the root of this:
\begin{equation}
    w^2+\frac{1w}{3}-\frac{2}{3}=(w+1)(w-2/3)=0
\end{equation}
So we have the solution for $w=-1$ and $w=2/3$.
Now $D$ has the following analytical solution:
\begin{equation}
D(t)=c_1t^{2/3}+c_2/t
\end{equation}

With our initial conditions of.
$D(1)=3,D'(1)=2$ we get $c_1=3,c_2=0$\\
$D(1)=10,D'(1)=-10$ $c_1=0,c_2=10$\\
$D(1)=5,D'(1)=0$ we get $c_1=3,c_2=2$

We also apply a numerical scheme to this ordinary differential equation, because some ODE's do not have analytical solutions, we use the fourth order Runge-Kutta method to tackle this problem. For this we go from a single second order differential equations to 2 first order differential equations which are coupled. We define $\frac{dD}{dt}=z(t)$
And then we get then as second equation for $z(t)$:
\begin{equation}
\frac{dz}{dt}=\frac{-4}{3t}z+\frac{2}{3t^2}D
\end{equation}

We apply the Runge-Kutta method to both ODE's and integrate from t=1 to t=1000, with a timestep of $\Delta t=0.01$year, for all 3 the initial values our Runge-Kutta method will be compared to the analytical solution.

%\lstinputlisting{exercise3.py}


%\lstinputlisting{exercise3.txt}


\begin{figure}
    \centering
    \subfloat{{\includegraphics[width=7cm]{plots/RK_1.png} }}%
    \qquad
    \subfloat{{\includegraphics[width=7cm]{plots/RK_2.png} }}%
    \caption{Two of our solutions of the Runge-Kutta method compared to the analytical solution.}
    \label{fig:rk}
\end{figure}

\begin{figure}[h]
   \centering
   \includegraphics[width=7cm]{plots/RK_3.png}
      \caption{Comparison of Runge-Kutta to the analytical solution}
\end{figure}


All of the Runge-Kutta methods work very well, there seems no mismatch of the Runge-Kutta method with the analytical solution. So this means that the fourth order already seems a very robust method, this is probably why this is the standard order.

\lstinputlisting{exercise3.py}
